%
% ARISE my SKALA antennas (will be chapter 7 in final product...)
%
\chapter{Future Prospects: SKALA Radio Antennas at Auger}\label{chap:skala}
The Pierre Auger Observatory hosts several prototype detectors and pathfinder air-shower arrays in addition to the detectors discussed in~\cref{chap:cosmicrays}.
This location greatly benefits prototyping new detectors because of the existing structure of the Auger observatory, allowing ease of access for detector maintenance and assistance from the local observatory staff when feasible.
Since 2022, 21 total SKALA (Square Kilometer Array Log-periodic Antenna) radio antennas have been deployed at Auger in the \SI{433}{\meter} spaced array of SD tanks.
These SKALA antennas are Log-Periodic Dipole Antennas (LPDAs), similar to the AERA LPDA antennas, and were originally designed for the Square Kilometer Array's low frequency telescope in Western Australia.
However, certain future cosmic-ray observatories with a surface radio component, such as IceCube-Gen2 and the Radio Neutrino Observatory Greenland (RNO-G), use SKALA antennas to reconstruct and/or veto cosmic ray air showers.
This chapter discusses deployment, hardware specifications, and preliminary analysis from SKALA antennas deployed at the Pierre Auger Observatory.

%through the cross-calibration of the new detectors to well trusted air-shower reconstructions.

%providing radio astronomy measurements from \SIrange{50}{350}{\mega\hertz}

%Combining electromagnetic and hadronic air-shower components is now moving into a new era of instrumentation based on current and upcoming air-shower observatories. The Square Kilometer Array (SKA) is a planned radio astronomy observatory located in Australia. It will include a dense array of radio antennas... \textbf{include more specifics here after reading papers}. One type of antenna planned to use in its construction is the Square Kilometer Array Log-periodic Antenna (SKALA) 


\section{An IceCube-Gen2 Prototype Station at the Pierre Auger Observatory}\label{sec:gen2prototype}

IceCube-Gen2 is a future upgrade of the existing IceCube Neutrino Observatory at the South Pole.
The upgrade plans to increase the volume of the current in-ice optical array from \SI{1}{\kilo\meter^{3}} to approximately \SI{8}{\kilo\meter^{3}}, in addition to deploying an in-ice radio array for neutrino detection and a surface array of scintillator panels and radio antennas for cosmic ray air shower detection.
The planned surface array will span nearly \SI{6}{\kilo\meter^{2}} to cover the full optical array extension, serving both as a veto to the optical array and to reconstruct cosmic ray air showers and their properties.
Each station of the surface array consists of eight scintillator panels, three SKALA radio antennas, and a central fieldhub hosting the local DAQ of the station.
A schematic of the surface station layout at the South Pole is shown in INSERTFIGURE.
A first prototype surface station was deployed at the South Pole in January 2020, with two more deployed in January 2025.
All South Pole prototypes have been mostly stable in their operations and data taking since deployment.
See INSERTREFERENCES for specifics regarding these stations.

In August 2022, a prototype surface station was deployed at the Pierre Auger Observatory.
The Auger prototype station has roughly the same design as the South Pole prototype stations, yet slightly different geometry and spacing due to the topology of the location (see INSERTDRONEPICTURE).
Other differences include the need for fencing around each detector position to prevent damage from roaming animals, and the lack of height adjustable stands, unnecessary in Argentina as constant snow accumulation is not an issue.
The prototype station is co-located in both the AERA footprint and the densest part of the SD array to serve the main science goal of coincident air-shower measurements with Auger detectors.

\subsection{Station Layout and Design}\label{sec:gen2layout}

Each arm of the Auger prototype station spans about \SI{70}{\meter} in length, with a pair of scintillator panels at the ends of each arm.
Another pair of scintillator panels lies at the station center, along with the station's local DAQ labeled as TAXI in INSERTLAYOUTSCHEMATIC. 
Halfway between the station center and the end of each arm is a SKALAv2 type radio antenna.
Each of the scintillator panels is connected to TAXI via a XXXX coaxial cable.
The SKALA antennas house two Low Noise Amplifiers (LNAs) at the 

have two perpendicular poarizations, where 


%reconstructions and vetoing air showers for the optical array.
%In addition to increasing the size of the current in-ice optical array for neutrino detection, the next generation upgrade of IceCube plans to expand the current surface array 