%
% ARISE my SKALA antennas (will be chapter 7 in final product...)
%
\chapter{Future Prospects: SKALA Radio Antennas at Auger}\label{chap:skala}
The Pierre Auger Observatory hosts several prototype detectors and pathfinder air-shower arrays in addition to the detectors discussed in~\cref{chap:cosmicrays}.
This location greatly benefits prototyping new detectors because of the existing structure of the Auger observatory, allowing ease of access for detector maintenance and assistance from the local observatory staff when feasible.
Since 2022, 21 total SKALA (Square Kilometer Array Log-periodic Antenna) radio antennas have been deployed at Auger in the \SI{433}{\meter} spaced array of SD tanks.
These SKALA antennas are Log-Periodic Dipole Antennas (LPDAs), similar to the AERA LPDA antennas, and were originally designed for the Square Kilometer Array's low frequency telescope in Western Australia.
However, future cosmic-ray observatories with a surface radio component, such as IceCube-Gen2 and the Radio Neutrino Observatory Greenland (RNO-G), use SKALA antennas to reconstruct and/or veto cosmic ray air showers.
This chapter discusses deployment, hardware specifications, and preliminary analysis from SKALA antennas deployed at the Pierre Auger Observatory.

%===============================
% Explain similarities for all SKALA antennas at Auger here? I.e. the DAQ, data transfer, laptop in the CRS, etc.
% Frank suggested doing this...
%===============================

%through the cross-calibration of the new detectors to well trusted air-shower reconstructions.

%providing radio astronomy measurements from \SIrange{50}{350}{\mega\hertz}

%Combining electromagnetic and hadronic air-shower components is now moving into a new era of instrumentation based on current and upcoming air-shower observatories. The Square Kilometer Array (SKA) is a planned radio astronomy observatory located in Australia. It will include a dense array of radio antennas... \textbf{include more specifics here after reading papers}. One type of antenna planned to use in its construction is the Square Kilometer Array Log-periodic Antenna (SKALA) 


\section{An IceCube-Gen2 Prototype Station at the Pierre Auger Observatory}\label{sec:gen2prototype}

IceCube-Gen2 is a future upgrade of the existing IceCube Neutrino Observatory at the South Pole.
The upgrade plans to increase the volume of the current in-ice optical array from \SI{1}{\kilo\meter^{3}} to approximately \SI{8}{\kilo\meter^{3}}, in addition to deploying an in-ice radio array for neutrino detection and a surface array of scintillator panels and radio antennas for cosmic-ray air shower detection.
The planned surface array will span nearly \SI{6}{\kilo\meter^{2}} to cover the full optical array extension, serving both as a veto to the optical array and to reconstruct cosmic-ray air showers and their properties.
Each station of the surface array consists of four pairs of scintillator panels, and three SKALA radio antennas arranged in a Mercedes-Benz like three-pointed star.
An elevated fieldhub, hosting the local DAQ of the station, is located at the center of the star and connects directly to the IceCube Lab (ICL) for station power, data transfer, and communications.
A general schematic of the surface station layout at the South Pole is shown in~\cref{fig:gen2_schematic_southpole}.
The first prototype surface station was deployed at the South Pole in January 2020, with two more deployed in January 2025.
All South Pole prototypes have been mostly stable in their operations and data taking since deployment.
See INSERTREFERENCES for specifics regarding these stations.

\begin{figure}
    \centering
    \includegraphics[trim={0cm 0cm 0cm 0cm},clip,width=0.7\textwidth]{chapter7/Gen2Prototype_SouthPoleSchematic.png}
    \caption[Prototype IceCube-Gen2 surface station schematic at the South Pole.]{The general schematic of a single prototype IceCube-Gen2 surface station deployed at the South Pole. Each station consists of eight scintillator panels, three SKALA radio antennas, and a central fieldhub hosting the local DAQ of the station. Figure taken from~\cite{gen2tdr}.}
    \label{fig:gen2_schematic_southpole} % Remove, no need to show schematic here if showing schematic for the actual station at Auger!
\end{figure}

In August 2022, a prototype surface station was deployed at the Pierre Auger Observatory.
~\cref{fig:augermap_gen2prototype} shows the location of the prototype station adjacent to the Central Radio Station (CRS), which hosts the AERA DAQ and communications systems.
The station is co-located in both the AERA footprint and the densest part of the SD array to serve the main science goals of coincident air-shower measurements with Auger detectors and a potential cross-calibration of the AERA radio antennas and SKALA antennas of the prototype station.

\begin{figure}
    \centering
    \includegraphics[trim={0cm 0cm 0cm 0cm},clip,width=0.6\textwidth]{chapter7/Gen2Prototype_ArrayPositionwCRS.pdf}
    \caption[Location of the IceCube-Gen2 prototype station at Auger.]{A map of the Auger SD array with an inset showing the location of the IceCube-Gen2 prototype station deployed at Auger. The location of the Central Radio Station of Auger is also shown. AERA stations, co-located in the inset region, are not shown due to space constrains. Figure taken from~\cite{icrc2025_verpoest}.}
    \label{fig:augermap_gen2prototype} % Include in first section of chapter, with locations of the ARISE antennas, ask Stef for code to make plot!
\end{figure}

\subsection{Station Design and Deployment}\label{sec:gen2design}

%The layout of the deployed station is shown in INSERTLAYOUTSCHEMATIC.
The Auger prototype station has roughly the same design as the South Pole prototype stations, yet slightly different geometry and spacing due to the topology of the location (see~\cref{fig:gen2_dronephoto}).
Other differences include the need for fencing around each detector position to prevent damage from roaming animals, and the lack of height adjustable stands, unnecessary in Argentina as constant snow accumulation is not an issue like at the South Pole.
Each arm of the Auger prototype station spans about \SI{70}{\meter} in length, with a pair of scintillator panels at the ends of each arm and another pair of scintillator panels at the station center, 
Halfway between the station center and the end of each arm is a SKALAv2 type radio antenna.
The SKALA antennas have two perpendicular polarizations, each with their own Low Noise Amplifier (LNA) housed in the top most part of the antenna, the trumpet; however, not all antennas in this prototype station have the same orientation (see~\cref{fig:gen2_schematic_auger}).
The radio LNAs are connected to the station DAQ, labeled as TAXI in~\cref{fig:gen2_schematic_auger}, from buried type-N coaxial cables.
Each coaxial cable is approximately \SI{40}{\meter} in length, with extra cable wrapped in a loop around the antenna legs, and connects to the LNA via a type-N to SMA cable \SI{1}{\meter} in length.

\begin{figure}
    \centering
    \includegraphics[trim={0cm 0cm 0cm 0cm},clip,width=0.5\textwidth]{chapter7/Gen2Prototype_DronePicture.JPG}
    \caption[Aerial drone image of the IceCube-Gen2 prototype station at Auger.]{Aerial drone image of the IceCube-Gen2 prototype station deployed at Auger. Paths were cleared to each detector by Se\~nor Porte using heavy machinery. The CRS is the large white shipping container at the bottom of the image. Nearby is the closest SD tank, Guili Jr. Drone image taken by Tim Huege.}
    \label{fig:gen2_dronephoto}
\end{figure}

\begin{figure}
    \centering
    \includegraphics[trim={0cm 0cm 0cm 0cm},clip,width=0.5\textwidth]{chapter7/Gen2Prototype_AugerSchematic_OUTDATED.pdf}
    \caption[Schematic of the deployed IceCube-Gen2 prototype station at Auger.]{The schematic of the IceCube-Gen2 prototype station deployed at Auger. All scintillator panels and radio antennas are labeled with their respective numbers. Both the locations of the station DAQ, labeled TAXI, and the CRS are shown, along with the direction of magnetic North. Schematic originally made by Marcos Cerda. ++++NEEDS UPDATES TO SCHEMATIC++++}
    \label{fig:gen2_schematic_auger}
\end{figure}

The scintillator panels connect to the DAQ from buried, custom made coaxial cables with 5-pin connectors on each end.
TAXI itself, the name of the DAQ electronics, has a metal housing and is stored in a weather proof aluminum box at the station center to protect it from the extreme weather conditions which can occur at the field in Argentina.
The aluminum box also provides weather protection for the cables ends connecting directly to TAXI.
While weather proof, the box can not prevent overheating of the electronics housed inside, with issues such as a freezing of the microDAQs (the DAQ system controlling the scintillators), heat related artifacts in the radio data, and thermal failure of the timing and communications board (White Rabbit Line Extension, i.e. WR-LEN, explained later) and its connecting optical fiber.
Each heat related issue has been resolved respectively by reprogramming the microDAQs, improving the artifact removal portion of the radio data analysis pipeline, and a full replacement of the WR-LEN board and its fiber.

Buried cables from the CRS provide power, data transfer, communications, and timing to TAXI.
The data transfer, communications, and timing of the station are all handled by a White Rabbit switch inside the CRS.
The White Rabbit switch takes a master time signal and distributes it amongst connected systems to provide sub-nanosecond timing accuracy and synchronization between multiple detectors.
The WR-LEN board, stored inside the TAXI housing, serves as a remote access point to the White Rabbit switch which is accessible over long distances using optical fibers.
A \SI{100}{\meter} buried military grade optical fiber provides this connection between the White Rabbit switch in the CRS and the WR-LEN board of the prototype station.
Another buried \SI{100}{\meter} cable from the CRS provides \SI{24}{\volt} DC power to TAXI.
Inside the CRS, a custom DC-DC power converter takes two \SI{12}{\volt} DC inputs and converts them into two \SI{24}{\volt} DC outputs to power both the White Rabbit switch and TAXI.
The power to both systems can be controlled from a network connection or with physical switches the custom power converter inside the CRS, allowing separate power cycling of the White Rabbit switch and TAXI.



%setting the timing of the TAXI electronics from a \SI{10}{\mega\hertz} and PPS signal and controlling the communications and data transfer of the TAXI 


A naked metal wire connects the metal TAXI housing to a buried copper rod and is used for grounding the TAXI electronics.
The dirt surrounding the buried rod was salted to increase soil conductivity and provide a better grounding.

%24V power for the DAQ and communications via a buried optical fiber come from the Central Radio Station (CRS).
%The Central Radio Station (CRS) provides 24V power to the DAQ from a buried power cable.


\subsection{Observation of the Galactic Center with SKALA Radio Antennas}\label{sec:gen2galactic}

Explain radio analysis pipeline (cleaning and filtering, etc.).
Determination of frequency bands with minimal noise contributions.
Subtraces method to determine RMS.
Fit moving average of RMS values.
%Include for all 3 frequency bands?


\begin{figure}
    \centering
    \includegraphics[trim={0cm 0cm 0cm 0cm},clip,width=0.7\textwidth]{chapter7/AugerStation_spectrum_800MHz_deconvolved.pdf}
    \caption[Radio frequency spectrum of the IceCube-Gen2 prototype station at Auger.]{Spectrum}
    \label{fig:gen2_rf_spectrum}
\end{figure}

% Subtraces plot from slide 23 of talk in Grand Rapids for IC collaboration meeting?

\begin{figure}
    \centering
    \includegraphics[trim={0cm 0cm 0cm 0cm},clip,width=0.8\textwidth]{chapter7/AugerStation_galactic.png}
    \caption[RMS time dependence of background radio signals.]{Time dependence of RMS}
    \label{fig:gen2_galactic}
\end{figure}

Table of fit results for all antennas in \SIrange{110}{130}{\mega\hertz}.

\subsection{First Coincident Air-Shower Observations with the Auger Surface Detector}\label{sec:gen2airshowers}



Air showers are measured by a coincidence requirement between scintillator triggers.

%reconstructions and vetoing air showers for the optical array.
%In addition to increasing the size of the current in-ice optical array for neutrino detection, the next generation upgrade of IceCube plans to expand the current surface array 

\section{ARISE: Auger Radio Infill SKALA Extension}\label{sec:arise}

The Auger Radio Infill SKALA Extension (ARISE) was deployed in March of 2025, centered upon SD tank Lety Jr. of the \SI{433}{\meter} spaced array of the Auger SD.
ARISE consists of six stations, with their centers organized in a hexagonal pattern surrounding Lety Jr.
Each station consists of three SKALA antennas and a central hub hosting the local station DAQ and a solar power system for the station.
The station centers are each located approximately \SI{75}{\meter} from Lety Jr., with the three station antennas deployed \SI{35}{meter} from the station center in a Mercedes-Benz like star pattern to have the same layout as the antennas of the IceCube-Gen2 prototype stations.
One arm of each station is aligned such that there is an antenna halfway between Lety Jr. and each station center.

Solar power system
Triggering from Lety Jr.
Lety Jr. has UMD also
Science potential of array through simulation study, full efficiency for vertical showers

