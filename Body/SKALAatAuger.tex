%
% ARISE my SKALA antennas (will be chapter 7 in final product...)
%
\chapter{Future Prospects: SKALA Radio Antennas at Auger}\label{chap:skala}
The Pierre Auger Observatory hosts several prototype detectors and pathfinder air-shower arrays in addition to the detectors discussed in~\Cref{chap:cosmicrays}.
This location greatly benefits prototyping new detectors because of the existing structure of the Auger observatory, allowing ease of access for detector maintenance and assistance from the local observatory staff when feasible.
Recently, several stations for radio detection of cosmic-ray air showers using Square Kilometer Array Log-periodic Antenna (SKALA) radio antennas have been deployed at Auger.
These SKALA antennas are Log-Periodic Dipole Antennas (LPDAs), similar to the AERA LPDA antennas, and were originally designed for the Square Kilometer Array's (SKA) low frequency telescope in Western Australia.
However, future cosmic-ray observatories with a surface radio component, such as IceCube-Gen2 and the Radio Neutrino Observatory Greenland (RNO-G), plan to use SKALA antennas to reconstruct and/or veto cosmic ray air showers.
This chapter discusses deployment, hardware specifications, and preliminary analysis from SKALA antennas deployed at the Pierre Auger Observatory.

%===============================
% Explain similarities for all SKALA antennas at Auger here? I.e. the DAQ, data transfer, laptop in the CRS, etc.
% Frank suggested doing this...
%===============================

%\section{Description of SKALA Stations, Data Acquisition Systems, and }
\section{The SKALA Detector System}\label{sec:skala_system}

\begin{figure}
    \centering
    \includegraphics[trim={0cm 0cm 0cm 0cm},clip,width=1.0\textwidth]{chapter7/AugerArray_SKALAInset.pdf}
    \caption[Map of all SKALA antennas at Auger.]{A map of the Auger SD array with an inset showing the location of all 21 SKALA antennas deployed at Auger. The location of the Central Radio Station of Auger is also shown. The antennas themselves are not to scale. AERA stations, co-located in the inset region, are not shown due to space constrains.}
    \label{fig:auger_map_skala_inset}
\end{figure}

Since 2022, 21 total SKALA-v2 radio antennas have been deployed at Auger in the \SI{433}{\meter} spaced array of SD tanks.
~\Cref{fig:auger_map_skala_inset} shows a map of all 21 SKALA antennas deployed in this region of the Auger array, inset against the full Auger array of SD tanks.
A single station for the SKALA detector system is formed by a group of three antennas arranged in a Mercedes-Benz like three-pointed star hence, in total, the 21 SKALA antennas at Auger constitute seven total stations.
One station forms the IceCube-Gen2 prototype station at Auger, while the other six stations form the Auger Radio Infill SKALA Extension (ARISE) array.
In~\Cref{fig:auger_map_skala_inset}, the IceCube-Gen2 prototype station is composed of the three SKALA antennas to the Northeast of the Auger Central Radio Station (CRS), shown as the orange star on the map, while the 18 SKALA antennas surrounding the Auger SD to the Northwest of the CRS form the six stations of ARISE. 
Stations for ARISE differ slightly from the IceCube-Gen2 prototype station, where these differences are described in more detail in the coming sections, one section each for the prototype station and ARISE.
This section serves to describe the similarities of the SKALA detector systems, specifically describing the hardware, data acquisition system (DAQ), and the data transfer and storage as this is mostly similar between the two types of station.

The backbone of all SKALA detector systems at Auger is housed inside the CRS, of which the main components are a laptop and a White Rabbit switch.
The laptop is an FZ-55 toughbook 


Inside the CRS, housed on an AERA DAQ rack, is the backbone of all SKALA stations at Auger, the White Rabbit switch.
It controls the data transfer, communications, and timing for each station via buried military grade optical fiber connections to the WR-LEN board at each SKALA station.
For timing, the White Rabbit switch takes a master time signal and distributes it amongst connected systems to provide sub-nanosecond timing accuracy and synchronization between multiple detectors.


%providing a deeper understanding of the hardware, data acquisition system (DAQ), and data transfer and storage for the coming sections.

The SKALA antennas themselves are all SKALA-v2 type antennas, shown in~\Cref{fig:skala_cad_drawing}, developed during antenna prototyping for the SKA-Low radio telescope.
SKALA antennas have two perpendicular polarizations, each with their own Low Noise Amplifier (LNA) housed in the top most part of the antenna, known as the antenna trumpet.
There are two types of LNA for SKALA-v2 antennas, a "top" LNA and a "bottom" LNA, referring to the LNA position inside the antenna trumpet with respect to its paired LNA.
Top and bottom LNAs are always paired together, one for each of the two perpendicular polarizations; however, no specific orientation for the polarization-LNA pairings are defined by default.
This must be defined and remain consistent between all antennas by the deployment team.
One deployed, a \SI{180}{\degree} rotation of the antenna will yield the same measured radio signal in principle due to a degeneracy in the pairing of the antenna design and LNA response.

\begin{figure}
    \centering
    \includegraphics[trim={0cm 0cm 0cm 0cm},clip,width=0.7\textwidth]{chapter7/skala_antenna_cad_drawing.png}
    \caption[CAD drawing of a SKALA-v2 radio antenna.]{A CAD drawing of a SKALA-v2 radio antenna. The antenna trumpet, at the top of the antenna, houses the two LNAs. The plastic square on the body of the antenna is called the antenna spider. The spider is used during antenna assembly to fix the arms perpendicularly and remains on the antenna body after deployment for stability. The SKALA antenna ends at the bottom black leg pieces. These pieces are the antenna legs and are used to fasten the antenna to a ground structure.}
    \label{fig:skala_cad_drawing}
\end{figure}

The metal arms of the LNAs are connected directly to the metal of each antenna arm and fixed in place with screws to ensure adequate contact between the antenna and LNA.
Adequate metal contact is necessary for the LNA to properly boost the weak signal measured by the SKALA antennas.
Without good contact, the radio signal has less amplification and is more prone to signal degradation in the coaxial cables on its travel to the DAQ electronics.
The LNA pairs are interlocked in the antenna trumpet (see~\Cref{fig:lna_lab_picture}) with a plastic spacer separating them to prevent crosstalk in the electronics.
Each LNA has a gold SMA connector used to transfer the radio signal post-LNA to the DAQ electronics, for the top LNA this SMA connector is visible in~\Cref{fig:lna_lab_picture}.
The signal from the LNA first passes through an SMA male to type-N female coaxial cable \SI{1}{\meter} in length, connected to a type-N male to type-N male coaxial cable of \SI{40}{\meter} length.
The antenna-opposite end of the \SI{40}{\meter} type-N cables connect directly to the station DAQ and are buried along the path between the antennas and DAQ to protect the cables from environmental factors, including roaming animals in the Argentinian Pampa such as cows, goats, and horses.

\begin{figure}
    \centering
    \includegraphics[trim={32cm 0cm 0cm 0cm},clip,width=0.6\textwidth,angle=270]{chapter7/LNAs_in_lab.jpg}
    \caption[Picture of a SKALA-v2 antenna trumpet with visible LNA.]{A picture of the SKALA-v2 antenna trumpet taken during the LNA assembly at the main Auger observatory campus in Malarg\"ue. The protective cap for the antenna trumpet has been removed to show the interlocked LNAs. Three of the four metal points where the LNA to antenna contact happens are visible, along with the top LNA and its golden SMA connector. The blue tape was used by the deployment team to track the antenna orientation after placement of the protective cap. In this image, the antenna orientation is such that the polarization arms connected to the top LNA point into and out of the page, while the polarization arms for the bottom LNA point left and right.}
    \label{fig:lna_lab_picture}
\end{figure}

The DAQ for each station of three SKALA antennas is based on the DAQ developed for the Transportable Array for eXtremely large area Instrumentation studies (TAXI) project and is hosted in a single metal container, also referred to as the TAXI housing.
The main DAQ components are handled by TAXI-v3.2, an updated version of the principal electronics board developed for the TAXI project.
As power, TAXI takes \SI{24}{\volt} DC power input through a custom 5-pin connector on the TAXI housing.
The \SI{24}{\volt} power is split into three separate DC-DC power converters of voltages \SI{3.3}{\volt}, \SI{4.5}{\volt}, and \SI{5}{\volt} to provide power to all TAXI components and all additional electronics.% connected to the TAXI-v3.2 board. 
These additional electronics boards connect directly to the TAXI-v3.2 board to complete the DAQ system (see~\Cref{fig:inside_taxi}).
The first additional electronics board is the radioTad, a front-end radio electronics board to pre-process the radio signals received from the LNAs before propagating the signals forward through to the DAQ.
The LNA signals are received at a type-N female connector of the TAXI housing and then are routed to the radioTad boards hosted internal to the housing via a coaxial cable.
Each radioTad board required \SI{4.5}{\volt} DC power, connects to a single SKALA antenna, and has two polarizations, labeled positive and negative, for the two antenna LNAs.
For simplicity, it was chosen upon deployment to have all top LNAs connect to the "positive" radioTad polarizations and all bottom LNAs to connect to the "negative" polarizations.
The radioTads power the LNAs using a bias-tee through the same coaxial cables with which they receive the LNA signals.
Once a signal reaches the radioTad from the LNA, both a high-pass and low-pass filter are applied to filter the frequency range within \SIrange{70}{350}{\mega\hertz}.
The filtered signals are converted to eight total differential signals for each LNA and fed to the TAXI board for further processing through a D-SUB 37 pin connector.
Inside the TAXI housing, the radioTads are covered with a custom made aluminum shielding to prevent bi-directional RFI leakage.

TAXI-v3.2 allows for two data taking modes to trigger the readout of the radio antennas, both controlled by the Field Programmable Gate Array (FPGA) of TAXI.
The first mode is a "soft" trigger mode, or a software initiated trigger, to readout the radio data.
In this mode, the FPGA sends a request to read the radio data every $X$ seconds, where $X$ is the FPGA clock time passed since the previous FPGA request.
This time interval $X$ is a programmable setting for the TAXI FPGA which can be configured through the communications channel to TAXI (to be described in the coming paragraphs).
The second mode is an external trigger mode.
In this case, an additional electronics board interfacing with an external detector is connected to TAXI and, upon fulfillment of a trigger condition set by the external detector electronics, sends the request to the TAXI FPGA to initiate the readout of the radio data.
The SKALA antennas for the IceCube-Gen2 prototype station and ARISE interface with different external detectors and will be described in their respective sections.
A more detailed description of the radioTad functionality and the TAXI-v3.2 processing of the radio signals is provided in ROXANNETHESIS.

Also connected to the TAXI board is the White Rabbit Line Extension (WR-LEN) board.
The WR-LEN board serves as a remote access point to the White Rabbit switch in the CRS, controlling the data transfer, communications, and timing of the station DAQ.
Buried military grade optical fibers connect the White Rabbit switch to the WR-LEN boards at each SKALA station, where the WR-LEN boards connect directly to the TAXI board via a Cat7 ethernet cable to control the communications and data transfer.
The Cat7 ethernet cables provide high-speed data transfer of approximately \SI{10}{Gbps}, allowing adequate data rates to transfer all data between TAXI and the White Rabbit switch, with minimal electromagnetic interference due to the cable shielding.
In addition, timing from the WR-LEN board is distributed to the TAXI board using two SMA cables, one to distribute the PPS signal and another for the \SI{10}{\mega\hertz} signal.
These timing signals originate at the White Rabbit switch in the CRS from a master timing signal.
\SI{5}{\volt} DC power is necessary for powering the WR-LEN board, coming directly from the TAXI internal DC-DC voltage converter.

\begin{figure}
    \centering
    \includegraphics[trim={0cm 0cm 0cm 0cm},clip,width=0.6\textwidth]{chapter7/InsideTAXIHousing_Picture.jpg}
    \caption[Picture inside of the TAXI housing.]{A picture inside the aluminum TAXI housing, specifically for the TAXI of the IceCube-Gen2 prototype station at Auger. The radioTad board is the exposed blue board in the lower half of the enclosure. Two additional radioTads are to the right of the exposed radioTad, but covered with their custom aluminum shielding. The WR-LEN board, in the internal TAXI housing configuration, is shown in the center on the upper aluminum rack of the housing. The green Cat7 ethernet cable connects directly to the TAXI-v3.2 electronics board sitting underneath all of the additional electronics. The eight red cables connect to an additional electronics board controlling the external detector used for triggering the radio read out of the IceCube-Gen2 prototype station. This electronics board will be described in further detail in~\Cref{sec:gen2prototype}.}
    \label{fig:inside_taxi}
\end{figure}

The WR-LEN board can be housed both internal or external to the TAXI housing.
When housed externally, the WR-LEN board can remain inside its original enclosure; however, modifications to the connectors of the TAXI housing and a custom power cable must be made in order to make all connections between WR-LEN and TAXI.
The externally housed WR-LEN setup provides greater heat dissipation compared to the internal setup, as the fully upwards facing surface of the original WR-LEN enclosure acts as a heat sink. 
When housed internal to the TAXI housing, the original enclosure for the WR-LEN does not fit inside the aluminum housing and must be removed from the board.
Instead, for heat dispersion, a heat sink is placed on the main Application Specific Integrated Circuit (ASIC) chip of the WR-LEN.
Summer temperature in Malarg\"ue can reach upwards of \SI{30}{\degreeCelsius}, with further heating from the Sun, while heat from the other electronics can increase the environmental temperature conditions surrounding the WR-LEN board to its upper limit for safe range of operation at \SI{50}{\degreeCelsius}.
Therefore a heat sink is necessary to mitigate malfunction of the WR-LEN when enclosed inside the TAXI housing.
For reference, thermal images using a FLIR infrared camera of WR-LEN boards operating outside of their original heat sink enclosure, without a heat sink on the main ASIC, and in normal lab conditions show temperature measurements of nearly \SI{100}{\degreeCelsius}.
These high temperatures also provoke further spikes and corruptions in the radio data measured by TAXI, hence the externally housed WR-LEN board is optimal for reducing these effects, yet at the expense of less dust protection.
As of now, the default for all SKALA stations is housing the WR-LEN board inside the TAXI housing.  



% Maybe I should stop here for this section...
Inside the CRS, housed on an AERA DAQ rack, is the backbone of all SKALA stations at Auger, the White Rabbit switch.
It controls the data transfer, communications, and timing for each station via buried military grade optical fiber connections to the WR-LEN board at each SKALA station.
For timing, the White Rabbit switch takes a master time signal and distributes it amongst connected systems to provide sub-nanosecond timing accuracy and synchronization between multiple detectors.

The WR-LEN board, stored inside the TAXI housing, serves as a remote access point to the White Rabbit switch which is accessible over long distances using optical fibers.
A \SI{100}{\meter} buried military grade optical fiber provides this connection between the White Rabbit switch in the CRS and the WR-LEN board of the prototype station.
Inside the CRS, a custom DC-DC power converter takes two \SI{12}{\volt} DC inputs and converts them into two \SI{24}{\volt} DC outputs to power both the White Rabbit switch and TAXI.
The power to both systems can be controlled from a network connection or with physical switches the custom power converter inside the CRS, allowing separate power cycling of the White Rabbit switch and TAXI.


%The LNA signals are received at the TAXI housing  a type-N female connector on the housing itself.
%Internal to the housing, this connection is then routed 


%on a single electronics board, also referred to as the TAXI board.
%All station DAQs for SKALA antennas in Argentina use TAXI-v3.2.

%Two additional electronics boards connect to the TAXI-v3.2 board, the radioTad board tasked with 


%The \SI{40}{\meter} type-N cables are buried along the path from the antennas


%to amplify the weak radio signals rece 




Describe antennas, LNAs, connectors of LNA and difference for top and bottom LNA. Describe TAXI and the triggering types. How TAXI interfaces with WR, power for TAXI, etc. Read into some of Roxanne's thesis for this. Discuss setup in CRS, transfer of data from CRS to Coihueco. Laptop in CRS.
Include pictures of antennas, LNAs, TAXI, WRS (from website).

%through the cross-calibration of the new detectors to well trusted air-shower reconstructions.

%providing radio astronomy measurements from \SIrange{50}{350}{\mega\hertz}

%Combining electromagnetic and hadronic air-shower components is now moving into a new era of instrumentation based on current and upcoming air-shower observatories. The Square Kilometer Array (SKA) is a planned radio astronomy observatory located in Australia. It will include a dense array of radio antennas... \textbf{include more specifics here after reading papers}. One type of antenna planned to use in its construction is the Square Kilometer Array Log-periodic Antenna (SKALA) 


\section{An IceCube-Gen2 Prototype Station at the Pierre Auger Observatory}\label{sec:gen2prototype}

IceCube-Gen2 is a future upgrade of the existing IceCube Neutrino Observatory at the South Pole.
The upgrade plans to increase the volume of the current in-ice optical array from \SI{1}{\kilo\meter^{3}} to approximately \SI{8}{\kilo\meter^{3}}, in addition to deploying an in-ice radio array for neutrino detection and a surface array of scintillator panels and radio antennas for cosmic-ray air shower detection.
The planned surface array will span nearly \SI{6}{\kilo\meter^{2}} to cover the full optical array extension, serving both as a veto to the optical array and to reconstruct cosmic-ray air showers and their properties.
Each station of the surface array consists of four pairs of scintillator panels, and three SKALA radio antennas arranged in a Mercedes-Benz like three-pointed star.
An elevated fieldhub, hosting the local DAQ of the station, is located at the center of the star and connects directly to the IceCube Lab (ICL) for station power, data transfer, and communications.
A general schematic of the surface station layout at the South Pole is shown in~\cref{fig:gen2_schematic_southpole}.
The first prototype surface station was deployed at the South Pole in January 2020, with two more deployed in January 2025.
All South Pole prototypes have been mostly stable in their operations and data taking since deployment.
See INSERTREFERENCES for specifics regarding these stations.

%\begin{figure}
%    \centering
%    \includegraphics[trim={0cm 0cm 0cm 0cm},clip,width=0.7\textwidth]{chapter7/Gen2Prototype_SouthPoleSchematic.png}
%    \caption[Prototype IceCube-Gen2 surface station schematic at the South Pole.]{The general schematic of a single prototype IceCube-Gen2 surface station deployed at the South Pole. Each station consists of eight scintillator panels, three SKALA radio antennas, and a central fieldhub hosting the local DAQ of the station. Figure taken from~\cite{gen2tdr}.}
%    \label{fig:gen2_schematic_southpole} % Remove, no need to show schematic here if showing schematic for the actual station at Auger!
%\end{figure}

In August 2022, a prototype surface station was deployed at the Pierre Auger Observatory.
~\cref{fig:augermap_gen2prototype} shows the location of the prototype station adjacent to the Central Radio Station (CRS), which hosts the AERA DAQ and communications systems.
The station is co-located in both the AERA footprint and the densest part of the SD array to serve the main science goals of coincident air-shower measurements with Auger detectors and a potential cross-calibration of the AERA radio antennas and SKALA antennas of the prototype station.

%\begin{figure}
%    \centering
%    \includegraphics[trim={0cm 0cm 0cm 0cm},clip,width=0.6\textwidth]{chapter7/Gen2Prototype_ArrayPositionwCRS.pdf}
%    \caption[Location of the IceCube-Gen2 prototype station at Auger.]{A map of the Auger SD array with an inset showing the location of the IceCube-Gen2 prototype station deployed at Auger. The location of the Central Radio Station of Auger is also shown. AERA stations, co-located in the inset region, are not shown due to space constrains. Figure taken from~\cite{icrc2025_verpoest}.}
%    \label{fig:augermap_gen2prototype} % Include in first section of chapter, with locations of the ARISE antennas, ask Stef for code to make plot!
%\end{figure}

\subsection{Station Design and Deployment}\label{sec:gen2design}

%The layout of the deployed station is shown in INSERTLAYOUTSCHEMATIC.
The Auger prototype station has roughly the same design as the South Pole prototype stations, yet slightly different geometry and spacing due to the topology of the location (see~\cref{fig:gen2_dronephoto}).
Other differences include the need for fencing around each detector position to prevent damage from roaming animals, and the lack of height adjustable stands, unnecessary in Argentina as constant snow accumulation is not an issue like at the South Pole.
Each arm of the Auger prototype station spans about \SI{70}{\meter} in length, with a pair of scintillator panels at the ends of each arm and another pair of scintillator panels at the station center, 
Halfway between the station center and the end of each arm is a SKALA-v2 type radio antenna.
The SKALA antennas have two perpendicular polarizations, each with their own Low Noise Amplifier (LNA) housed in the top most part of the antenna, the trumpet; however, not all antennas in this prototype station have the same orientation (see~\cref{fig:gen2_schematic_auger}).
The radio LNAs are connected to the station DAQ, labeled as TAXI in~\cref{fig:gen2_schematic_auger}, from buried type-N coaxial cables.
Each coaxial cable is approximately \SI{40}{\meter} in length, with extra cable wrapped in a loop around the antenna legs, and connects to the LNA via a type-N to SMA cable \SI{1}{\meter} in length.

\begin{figure}
    \centering
    \includegraphics[trim={0cm 0cm 0cm 0cm},clip,width=0.5\textwidth]{chapter7/Gen2Prototype_DronePicture.JPG}
    \caption[Aerial drone image of the IceCube-Gen2 prototype station at Auger.]{Aerial drone image of the IceCube-Gen2 prototype station deployed at Auger. Paths were cleared to each detector by Se\~nor Porte using heavy machinery. The CRS is the large white shipping container at the bottom of the image. Nearby is the closest SD tank, Guili Jr. Drone image taken by Tim Huege.}
    \label{fig:gen2_dronephoto}
\end{figure}

%\begin{figure}
%    \centering
%    \includegraphics[trim={0cm 0cm 0cm 0cm},clip,width=0.5\textwidth]{chapter7/Gen2Prototype_AugerSchematic_OUTDATED.pdf}
%    \caption[Schematic of the deployed IceCube-Gen2 prototype station at Auger.]{The schematic of the IceCube-Gen2 prototype station deployed at Auger. All scintillator panels and radio antennas are labeled with their respective numbers. Both the locations of the station DAQ, labeled TAXI, and the CRS are shown, along with the direction of magnetic North. Schematic originally made by Marcos Cerda. ++++NEEDS UPDATES TO SCHEMATIC++++}
%    \label{fig:gen2_schematic_auger}
%\end{figure}

The scintillator panels connect to the DAQ from buried, custom made coaxial cables with 5-pin connectors on each end.
TAXI itself, the name of the DAQ electronics, has a metal housing and is stored in a weather proof aluminum box at the station center to protect it from the extreme weather conditions which can occur at the field in Argentina.
The aluminum box also provides weather protection for the cables ends connecting directly to TAXI.
While weather proof, the box can not prevent overheating of the electronics housed inside, with issues such as a freezing of the microDAQs (the DAQ system controlling the scintillators), heat related artifacts in the radio data, and thermal failure of the timing and communications board (White Rabbit Line Extension, i.e. WR-LEN, explained later) and its connecting optical fiber.
Each heat related issue has been resolved respectively by reprogramming the microDAQs, improving the artifact removal portion of the radio data analysis pipeline, and a full replacement of the WR-LEN board and its fiber.

Buried cables from the CRS provide power, data transfer, communications, and timing to TAXI.
The data transfer, communications, and timing of the station are all handled by a White Rabbit switch inside the CRS.
The White Rabbit switch takes a master time signal and distributes it amongst connected systems to provide sub-nanosecond timing accuracy and synchronization between multiple detectors.
The WR-LEN board, stored inside the TAXI housing, serves as a remote access point to the White Rabbit switch which is accessible over long distances using optical fibers.
A \SI{100}{\meter} buried military grade optical fiber provides this connection between the White Rabbit switch in the CRS and the WR-LEN board of the prototype station.
Another buried \SI{100}{\meter} cable from the CRS provides \SI{24}{\volt} DC power to TAXI.
Inside the CRS, a custom DC-DC power converter takes two \SI{12}{\volt} DC inputs and converts them into two \SI{24}{\volt} DC outputs to power both the White Rabbit switch and TAXI.
The power to both systems can be controlled from a network connection or with physical switches the custom power converter inside the CRS, allowing separate power cycling of the White Rabbit switch and TAXI.



%setting the timing of the TAXI electronics from a \SI{10}{\mega\hertz} and PPS signal and controlling the communications and data transfer of the TAXI 


A naked metal wire connects the metal TAXI housing to a buried copper rod and is used for grounding the TAXI electronics.
The dirt surrounding the buried rod was salted to increase soil conductivity and provide a better grounding.

%24V power for the DAQ and communications via a buried optical fiber come from the Central Radio Station (CRS).
%The Central Radio Station (CRS) provides 24V power to the DAQ from a buried power cable.


\subsection{Observation of the Galactic Center with SKALA Radio Antennas}\label{sec:gen2galactic}

Explain radio analysis pipeline (cleaning and filtering, etc.).
Determination of frequency bands with minimal noise contributions.
Subtraces method to determine RMS.
Fit moving average of RMS values.
%Include for all 3 frequency bands?


\begin{figure}
    \centering
    \includegraphics[trim={0cm 0cm 0cm 0cm},clip,width=0.7\textwidth]{chapter7/AugerStation_spectrum_800MHz_deconvolved.pdf}
    \caption[Radio frequency spectrum of the IceCube-Gen2 prototype station at Auger.]{Spectrum}
    \label{fig:gen2_rf_spectrum}
\end{figure}

% Subtraces plot from slide 23 of talk in Grand Rapids for IC collaboration meeting?

\begin{figure}
    \centering
    \includegraphics[trim={0cm 0cm 0cm 0cm},clip,width=0.8\textwidth]{chapter7/AugerStation_galactic.png}
    \caption[RMS time dependence of background radio signals.]{Time dependence of RMS}
    \label{fig:gen2_galactic}
\end{figure}

Table of fit results for all antennas in \SIrange{110}{130}{\mega\hertz}.

\subsection{First Coincident Air-Shower Observations with the Auger Surface Detector}\label{sec:gen2airshowers}



Air showers are measured by a coincidence requirement between scintillator triggers.

%reconstructions and vetoing air showers for the optical array.
%In addition to increasing the size of the current in-ice optical array for neutrino detection, the next generation upgrade of IceCube plans to expand the current surface array 

\section{ARISE: Auger Radio Infill SKALA Extension}\label{sec:arise}

The Auger Radio Infill SKALA Extension (ARISE) was deployed in March of 2025, centered upon SD tank Lety Jr. of the \SI{433}{\meter} spaced array of the Auger SD.
ARISE consists of six stations, with their centers organized in a hexagonal pattern surrounding Lety Jr.
The station centers are located approximately \SI{80}{\meter} from Lety Jr., each consisting of three SKALA-v2 antennas arranged in a Mercedes-Benz like star pattern with each antenna \SI{40}{\meter} from the station center. 
One arm of each station is aligned such that an antenna is halfway between Lety Jr. and each station center.
The final positions of the 18 SKALA-v2 antennas were optimized through a simulation study to maximize the number of vertical air showers above the threshold for full reconstruction efficiency.
For this array of 18 SKALA antennas, a larger overall footprint will observe more air showers, but will increase the energy threshold for full efficiency due to the increased spacing between antennas. 
With the described antenna positions, full reconstruction efficiency ($>$ 97\%) is reached at \SI{300}{\peta\electronvolt} for near-vertical air showers ($\theta <$ \SI{30}{\degree}), resulting in reconstruction statistics of $\simeq$\,45 air showers per year~\cite{uhecr2024_verpoest}.
Current machine learning based radio denoising and reconstruction algorithms can lower this energy threshold by a factor of two once implemented in the data pipeline, increasing the statistics of reconstructed air showers~\cite{ABDUL_THESIS}.
ARISE
Radio reconstructions of near-vertical air showers at Auger now, for the first time, have full reconstruction efficiency above \SI{300}{\peta\electronvolt} within the footprint of ARISE.

%ARISE is the first radio array of its kind for air shower detection at Auger with a studied threshold for full efficiency.

%The science potential of ARISE is manifold. In addition to fully efficient air-shower reconstructions, the co-location of ARISE within the UMD and AERA footprints of Auger 
%Combining ARISE radio detection with UMD measurements provide a direct way to simultaneously reconstruct $X_{\rm max}$ and $N_{\mu}$ for the same air showers, as previous analyses with SKALA-v2 antennas have proven their reconstruction potential for $X_{\rm max}$ (CITE XMAX ICECUBE ROXANNE).

In addition to fully efficient air-shower reconstructions, the science potential of ARISE is manifold, as it is co-located within the UMD and AERA footprints of Auger.
Combining ARISE radio detection with UMD measurements provide a direct way to simultaneously reconstruct $X_{\rm max}$ and $N_{\mu}$ for the same air showers, as previous analyses with SKALA-v2 antennas have proven their reconstruction potential for $X_{\rm max}$ (CITE XMAX ICECUBE ROXANNE).
ARISE can then serve as a pathfinder array for this hybrid detection method, making progress towards major science goals of next-generation air-shower arrays in the quest for constraining high-energy hadronic interaction models and mass composition measurements.
Furthermore, the co-location with AERA allows a potential cross-calibration to be performed between the AERA and SKALA radio energy scales.
Given the currently foreseen runtime of Auger until 2035, approximately 500 air showers will be reconstructed with ARISE, providing a large dataset of air showers for high-quality analyses with intermittent results in the coming years.


%Previous analyses with SKALA-v2 antennas have shown $X_{\rm max}$ reconstruction is feasible (CITE XMAX ICECUBE), therefore combining ARISE radio reconstructions with UMD measurements provide a direct way to combine  
%The UMD provides a direct measurement of the muonic component of air showers, while the 


%science goals of next-generation air-shower arrays in the quest for constraining high-energy hadronic interaction models and mass composition measurements.


TAXI, the same DAQ as the IceCube-Gen2 prototype station, is used as the local DAQ for each station and is housed inside a weather proof box at the local station center.
Communications to TAXI are performed via buried optical fiber cables from the station centers to the White Rabbit switch in the CRS, in the same way as the IceCube-Gen2 prototype station. 
The readout of the radio data is triggered by the local station T1 trigger of the Auger SD tank Lety Jr (see section XXX for an explanation of Auger SD triggers).
The software-level T1 trigger is routed through a \SI{50}{\ohm} terminated SMA cable from the TRIGGEROUT (input as code) port of the Auger UUB to a custom-made trigger splitter housed inside the electronics dome of Lety Jr.
The custom designed aluminum box splits the electrical signal from the TRIGGEROUT (input as code) connector six ways, each feeding into an optical Tx, digital-to-optical signal converter, board.
Six \SI{100}{\meter} military grade optical fibers are trenched from Lety Jr. to each of the six ARISE station centers, connecting the optical Tx boards to an optical Rx board inside the TAXI enclosure.
The optical Rx board reconverts the signal from optical back to digital for TAXI to initiate the readout of the radio signal by each of the station's three SKALA antennas.
In principle, the optical trigger signals should arrive at each TAXI simultaneously; however, nanosecond to sub-nanosecond time differences could occur due to the conversion and reconversion process (Check actual value with Stef/Carmen).

Each of the TAXIs is autonomously powered from a solar power system at the station center.
The solar power system consists of six batteries, a solar panel, and a charge controller to control the charging of the batteries and regulate the voltage supplied to TAXI.
Six \SI{12}{\volt} batteries are used per station, connected in a three series two parallel configuration to provide the \SI{24}{\volt} necessary to power TAXI.
The batteries are also stored in the station center at the bottom of the wooden box and separated from the TAXI and charge controller via a wooden plank separator.
Station centers are equipped with a dual-sided solar panel of type (INPUT SP TYPE) to charge the batteries during good weather.
Solar panels are aligned with the broad side perpendicular with compass North to maximize the amount of sunlight impacting the solar cells of the panels.

Connected in series between the charge controller and TAXI inside the station center is the ARISE Control and Housekeeping (CHK) box.
The CHK box serves two purposes: 1. to allow remote power cycling of the TAXI, and 2. records the voltage and current supplied to TAXI.
The heart of the CHK box is a BeagleBone Black (BBB) single-board debian computer which handles all operations using python and bash scripts to remotely turn on/off TAXI and control the power logging service.
Connection to BBB is set from a Ubiquiti LiteBeam antenna the connects to a Ubiquiti Rocket housed at the CRS.
The 24V powering TAXI is used to also power the LiteBeam antenna from a power over ethernet (PoE) converter.


%Once connected to BBB, shell scripts can be executed to remotely turn on/off TAXI and start the power logging service.
%All functions of the CHK box are handled by a BeagleBone Black single-board linux computer.

Communications, data transfer, and timing of the TAXIs are all provided by buried military grade optical fibers trenched from the CRS to Lety Jr., and then to each ARISE station center along each arm of the array.
This protocol and setup is the same as for the IceCube-Gen2 prototype station, except for in one of the ARISE stations where the WR-LEN board is housed outside of the TAXI housing itself and inside its original enclosure.
The externally housed WR-LEN will reduce heat inside the TAXI housing as the original WR-LEN enclosure contains a heat sink over its full surface, allowing the heat to dissipate more efficiently in this configuration rather than within the TAXI housing and only with a small heat sink attached to the hottest FPGA chip of the WR-LEN.
The WR-LEN temperatures can exceed \SI{80}{\degreeCelsius} in the field at Argentina because of the warm weather environment and heating from the Sun, therefore reducing the heat inside the TAXI enclosure is a safer environment for the electronics.
%The original WR-LEN enclosure contains a heat sink over its full surface

The T1 trigger that controls the readout of the radio signal at each station arrives at TAXI from Lety Jr. at an average rate of \SI{100}{\hertz}.
Each radio binary file has a storage size of approximately \SI{100}{\mega\byte} resulting in a daily data volume of \SI{2}{\tera\byte} from ARISE; yet, in the current setup, data storage is limited to an \SI{8}{\tera\byte} SSD connected to the Fz-55 toughbook in the CRS.
The T1 level trigger contains mostly low-energy air showers to be used for calibrating the PMTs.
Therefore applying a match of the T1 timestamps at TAXI with the first physics level trigger (T3) timestamps of events in which Lety Jr. participates results in a lowering of the rate to $\mathcal{O}$(\SI{0.01}{\hertz}).
Matching of events is done locally at the toughbook laptop in the CRS by a daily transfer of the Auger iklog files to the laptop and then matching by timestamp with each radio binary.
Once matched the files are then transferred to a storage computer at the Karlsruhe Institute for Technology, and then further transferred to a data storage computer at the University of Delaware.

The T1 trigger rate averages to \SI{100}{\hertz}, yet is a Poissonian process.
Therefore, at any given moment the rate can reach values $>$\SI{500}{\hertz} that introduce an artificial deadtime in the radio data readout due to a buffer pileup at TAXI.
Currently, this results in a deadtime of approximately 50\% in the radio data taking.
Firmware updates for TAXI are being developed that will improve TAXI's handling of large data rates and in hand decrease the length of data taking deadtime.

%Currently, the large trigger rate of \SI{100}{\hertz} that reaches each TAXI introduces an artificial deadtime in the radio data readout due to a buffer pileup.

%Currently, the ARISE data is saved on an \SI{8}{\tera\byte} SSD connected to the Fz-55 toughbook in the CRS which handles all data transfer and storage

%received at each TAXI from Lety Jr. arrives at a \SI{100}{\hertz} rate

%Within ARISE, the T1 trigger received from the Lety Jr. arrives 


Additional communication to the TAXIs comes from a Ubiquiti Rocket and LiteBeam system.
This 

Beagle Bone Black


%Inside the station center a charge controller is used to regulate the voltage between  

%Each of the six station centers is solar power system for the station.



%The optically converted signal from the Tx board travels from 
%Once the local station receives the T1 



Solar power system
Triggering from Lety Jr.
Lety Jr. has UMD also
Science potential of array through simulation study, full efficiency for vertical showers

%Each station consists of three SKALA antennas and a central hub hosting the local station DAQ and a solar power system for the station.
%The station centers are each located approximately \SI{75}{\meter} from Lety Jr., with the three station antennas deployed \SI{35}{meter} from the station center in a Mercedes-Benz like star pattern to have the same layout as the antennas of the IceCube-Gen2 prototype stations.

% Overlap with AERA is only with the Dutch stations, i.e. the non-functioning part of the array
% Potential for comparing the energy scales indirectly from the SD.



% List of plots to include here:
% 1. Map of array w/ all SKALA antennas
% 8. Network setup at CRS

% 2. Map of IceCube station w/ scintillators and antennas, similar to Marcos' map but needs updates, include antenna orientation

% 4. ARISE Station center schematic
% 3. ARISE CHK box schematic
% x. Inside station center (photo)
% 5. Schematics showing fiber connections (see talk)
% 6. Inside of TAXI (photo)
% 7. Trigger splitter (photo)
% x. Photo while debugging LiteBeam connection
% 9. Background spectra of all ARISE antennas

% Go through photos from deployment and pick ones which are relevant, past talks also...